\documentclass[a4paper, 10pt]{article}
\usepackage{standart-header}
\usepackage{textcomp}
\fancyhead[R]{Винер Даниил. БЭАД232}
\fancyfoot [C] {\thepage}
\usepackage{pgfplots}
\setlength{\parindent}{15pt}
\setlength{\parskip}{2mm}
% \setlist[itemize]{left=1cm}
% \setlist[enumerate]{left=1cm}
\fancyhead[L]{ТВиМС—1. Семинар 04.10.2024}

\begin{document}
\definition Последовательность случацных величин  $(X_n)_{n=1}^{\infty}$ сходится по вероятности к случайной величине $X$, если 
\begin{equation*}
    \forall\varepsilon>0\ \lim\limits_{n\to\infty}\prob{|X_n-X|>\varepsilon}=0
\end{equation*}

Обозначение: $X_n\overset{\mathbb{P}}{\longrightarrow}X$ при $n\to\infty$

Свойства сходимости по вероятности:
\begin{enumerate}
    \item Если $X_n\byprob a,Y_n\byprob b$, то $X_n+Y_n\byprob a+b$
    \item Если $X_N$
\end{enumerate}

\theorem  (Слуцкого) Пусть $X_n\byprob a$ и $g(x)$ — непрерывная функция в точке $a$, тогда $g(X_n)\byprob g(a)$

\theorem Пусть $X_n\byprob a, Y_n\byprob b$ и $g(x,y)$ — непрерывная в точке $(x,y)=(a,b)$, тогда $g(X_n,Y_n)\byprob g(a,b)$

\theorem Закон больщих чисел. Пусть $(X_n)_{n=1}^{\infty}$ — последовательность одинаково распределенных с конечным математическим ожиданием, тогда 
\begin{equation*}
    \overline{X_n}\byprob \matwait{X_i}\text{ при }n\to\infty\text{, где }\overline{X_n}=\frac{X_1+\ldots+X_n}{n}
\end{equation*}

\section*{Задача 1.2}


\theorem (достаточное условие сходимости по вероятности)

Пусть выполнено
\begin{enumerate}
    \item $\matwait{X_n}\to a$
    \item $\dispersia{X_n}\to0$
\end{enumerate}

Тогда, $X_n\byprob a$








\end{document}