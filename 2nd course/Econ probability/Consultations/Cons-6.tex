\documentclass[a4paper, 10pt]{article}
\usepackage{standart-header}
\fancyhead[R]{Винер Даниил. БЭАД232}
\fancyfoot [C] {\thepage}
\setlength{\parindent}{15pt}
\setlength{\parskip}{2mm}
\newcommand{\orr}{\text{ or }}
\newcommand{\andd}{\text{ and }}
% \newcommand{\prob}[1]{\mathbb{P}\left(\left\{#1\right\}\right)}
% \newcommand{\matwait}[1]{\mathbb{E}\left[#1\right]}
% \newcommand{\dispersia}[1]{\mathbb{D}\left[#1\right]}
\fancyhead[L]{ТВиМС—1. Консультация—6 08.11.2024}

\begin{document}
На этой консультации повторяли материал из \href{https://docs.yandex.ru/docs/view?url=ya-disk-public%3A%2F%2Fc8XCsejPj%2FtKt87zMG2T7%2BKeFtYtnJjbzGHtYx%2B1a05z5eJWZ7rUsYBXdDhpwau2sLK2WbwBkR%2F%2FqfmVHoPilw%3D%3D%3A%2FЛИСТКИ%2FМОДУЛЬ-1%2FЛИСТОК-5%20%5B2024--2025%5D.pdf&name=ЛИСТОК-5%20%5B2024--2025%5D.pdf}{листка №5 из модуля 1}

\definition Пусть задано измеримое пространство $(\Omega,\fk)$. Функция $\xi:\Omega\longrightarrow\mathbb{R}$, $\xi=\xi(\omega)$, $\omega\in\Omega$ называется \textit{измеримой} функцией относительно \s-алгебры $\fk$, если $$\forall c\in\mathbb{R}\ \{\omega\in\Omega:\xi(\omega)>c\}\in\fk$$

Назовем условием измеримости функцию, измеримую относительно \s-алгебры $\fk$, также называемую $\fk$-измеримой функцией или функцией, согласованной с \s-алгеброй $\fk$
    
\section*{№3}
$(\Omega, \fk)$ — измеримое пространство, $\xi:\Omega\rightarrow\mathbb{R}$ — $\fk$-измеримая функция $\forall c\in\mathbb{R}$

Доказать утверждения
\begin{itemize}
    \item[\textbf{а)}] $\underbrace{\{\omega\in\Omega:\ \xi(\omega)\geqslant c\}}_{=LHS}=\underbrace{\displaystyle\bigcap_{i=1}^n \left\{\omega\in\Omega:\ \xi(\omega)>c-\frac{1}{n}\right\}\in\fk}_{=RHS}$
    \begin{itemize}
        \item $(LHS\subseteq RHS)$: $\omega_0\in LHS\Longrightarrow\xi(\omega)\geqslant c\Longrightarrow\forall n\in\mathbb{N}:  \xi{w_0}>c-\frac{1}{n}\Longrightarrow \omega_0\in RHS$
        \item $(RHS\subseteq LHS)$: $\omega_0\in RHS\Longrightarrow\forall n\in\mathbb{N}: \xi(\omega_0)>c-\frac{1}{n}\Longrightarrow\xi(\omega_0)\geqslant\lim\limits_{n\mapsto\infty}\left(c-\frac{1}{n}\right)=c\Longrightarrow\omega_0\in LHS$
    \end{itemize}
    \item[\textbf{b)}] $\{\omega\in\Omega:\xi(\omega)=c\}=\underbrace{\{\omega\in\Omega:\xi(\omega)\geqslant c\}}_{\in\fk\text{(см.п.а)}}\setminus\underbrace{\{\omega\in\Omega:\xi(\omega)>c\}}_{\in\fk\text{(по опр. $\fk$-изм.ф-и)}}\in\fk$
    \item[\textbf{c)}] $\{\omega\in\Omega:\xi(\omega)\leqslant c\}=\underbrace{\Omega}_{\in\fk}\setminus\underbrace{\{\omega\in\Omega:\xi(\omega)>c\}}_{\in\fk\text{(по опр. $\fk$-изм.ф-и)}}\in\fk$ 
    \item[\textbf{d)}] $\{\omega\in\omega:\xi(\omega)<c\}=\underbrace{\{\omega\in\Omega:\xi(\omega)\leqslant c\}}_{\in\fk\text{(см. п.c)}}\setminus\underbrace{\{\omega\in\Omega:\xi(\omega)=c\}}_{\in\fk\text{(см.п.b)}}$
\end{itemize}


\section*{№4}
$(\Omega,\fk)$ — измеримое пространство, $\xi:\Omega\rightarrow\mathbb{R}$ — $\fk$-измеримая функция

Докажите, что $\xi^2(\omega)$ — $\fk$-измерима
Рассмотрим два случая
\begin{itemize}
    \item $c<0:\ \{\omega\in\Omega:\xi^2(\omega)>c\}=\Omega\in\fk$
    \item $c\geqslant 0:\ \{\omega\in\Omega:\xi^2(\omega)>c\}=\underbrace{\{\omega\in\Omega:\xi(\omega)<-\sqrt{c}\}}_{\in\fk\text{(см. п.3d)}}\cup\underbrace{\{\omega\in\Omega:\xi(\omega)>\sqrt{c}\}}_{\in\fk\text{(по опр.$\fk$-изм.ф-и)}}\in\fk$
\end{itemize}

\section*{№5}
% TODO

\section*{№6}



\end{document}