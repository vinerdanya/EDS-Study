\documentclass{article}
\usepackage{header} 

\title{\LARGE{Теория вероятностей и математическая статистика—1}\\
Консультация по максимуму—2}
\author{irreitman, \href{https://t.me/danya_vin}{@danya\_vin}}
\date{Версия от \today} 

\begin{document}

\maketitle
\begin{flushright}
    \textit{Яна, спасибо за мак}
\end{flushright}
\noindent Все задания взяты из максимума—2 ИП 2023-2024

\section*{№2}
\begin{tcolorbox}[colback=blue!20!white, colframe=black!100!black]
    Величины $X$ и $Y$ независимы. Случайная величина $Y$ равномерно распределена на отрезке $[0,1]$. Случайная величина $X$ имеет плотность распределения
    \begin{equation*}
        f_X(x)=\begin{cases}
        2 x,&\text {если } x \in[0 ; 1] \\
        0,&\text {иначе }
        \end{cases}
    \end{equation*}
    \begin{enumerate}
        \item[a)] Найдите функцию плотности суммы $f_{X+Y}(z)$
        \item[b)] Найдите $\mathbb{E}(X+Y)$
        \item[c)] Найдите интерквартильный размах величины $X$
    \end{enumerate}
\end{tcolorbox}
\begin{enumerate}
    \item[a)] $f_{X+Y}=\displaystyle\int\limits_{-\infty}^{+\infty} f_{X}(x-U)f_{Y}(U)\d{u}$

    Отметим, что $f_{Y}=\begin{cases}
        1,&\text{если }x\in[0;1]\\
        0,&\text{иначе}
    \end{cases}$
    
     Чтобы мы могли взять интеграл нужно, чтобы $\begin{cases}
        (x-U)\in[0;1]\\
        U\in[0;1]
    \end{cases}\Longrightarrow\begin{cases}
        U\in[x-1;x]\\
        U\in[0;1]
    \end{cases}$

    Тогда, у нас есть два случая:
    \begin{itemize}
        \item $\begin{cases}
            x\geqslant 0\\
            x-1<0
        \end{cases}\Longrightarrow x\in[0;1)$

        Тогда, интеграл будет равен $\displaystyle\int\limits_0^x 2(x-U)\d{u}=2x^2-x^2=x^2$

        \item $\begin{cases}
            x\geqslant 1\\
            x-1\leqslant1
        \end{cases}\Longrightarrow x\in[1;2]$

        Тогда, интеграл равен $\displaystyle\int\limits_{x-1}^{1}2(x-U)\d{u}=2x-x^2$
    \end{itemize}

    Таким образом, $f_{X+Y}=\begin{cases}
        x^2,&x\in[0;1)\\
        2x-x^2,&x\in[1;2]\\
        0,&\text{иначе}
    \end{cases}$

    \item[b)] $\matwait{X+Y}=\matwait{X}+\matwait{Y}=\matwait{X}+\frac{1}{2}$

    $f_{X}=\begin{cases}
        2x,&x\in[0;1]\\
        0,&\text{иначе}
    \end{cases}$

    Тогда, $\matwait{X}=\displaystyle\int\limits_0^1 xf_{X}(x)\d{x}=\int\limits_0^1 2x^2\d{x}=\frac{2}{3}$

    Итого, $\matwait{X+Y}=\matwait{X}+\frac{1}{2}=\frac{2}{3}+\frac{1}{2}=\frac{7}{6}$
    \item[c)] Найти $IR=q_{0,75}-q_{0,25}$

    $f_{X}=\begin{cases}
        2x,&x\in[0;1]\\
        0,&\text{иначе}
    \end{cases}\Longrightarrow F_{X}=\begin{cases}
        0,&x<0\\
        x^2,&x\in[0;1]\\
        1,&x>1
    \end{cases}$

    \definition Квантиль уровня $a$ — корень уравнения $F_{X}(q_{a})=a$

    Тогда, $\begin{cases}
        q_{0,25}: x^2=0,25\Longrightarrow x=0,5\\
        q_{0,75}: x^2=0,75\Longrightarrow x=0,5\sqrt{3}
    \end{cases}$ и $IR=q_{0,75}-q_{0,25}=0,5(\sqrt{3}-1)$
    
\end{enumerate}


\newpage
\section*{№3}
\begin{tcolorbox}[colback=blue!20!white, colframe=black!100!black]
    Сибирский крокодил Утундрий решил заняться риск-менеджментом. В этот раз он изучает теорию эффективных портфелей ценных бумаг. Величины $R_1$ и $R_2$ — это доходности двух рисковых ценных бумаг с $\matwait{R_1}=\matwait{R_2}=0.15, \dispersia{R_1}=0.49, \dispersia{R_2}=1$ и $\operatorname{cov}\left(R_1, R_2\right)=-0.35$.
    
    Крокодил Утундрий знает, что:
    \begin{itemize}
        \item Доходность портфеля вычисляется по формуле $R=w_1 R_1+w_2 R_2$, где $w_1 \geqslant 0$ и $w_2 \geqslant 0$ — это доли первой и второй ценный бумаг в портфеле, соответственно, и $w_1+w_2=1$
        \item Ожидаемая доходность портфеля равна $\matwait{R}$
        \item Квадратичный риск портфеля определяется как $\dispersia{R}$
    \end{itemize}
    Утундрий составил два портфеля $A$ и $B$. Доли ценных бумаг, с которыми ценные бумаги входят в портфель $A$, заданы вектором $w^A=(0.7,0.3)$, а в портфель $B-$ вектором $w^B=(1,0)$

    Помогите Утундрию выполнить следующие задания:
    \begin{enumerate}
        \item[a)] Найдите ожидаемые доходности портфелей $A$ и $B$
        \item[b)] Какой из портфелей $A$ или $B$ предпочтительнее, если Утундрий является рискофобом, то есть не приемлет риск?
        \item[c)] Найдите ковариацию доходностей портфелей $A$ и $B$
        \item[d)] Составьте портфель $C$, который имеет наименьший квадратичной риск среди всех портфелей с ожидаемой доходностью $0.15$   
    \end{enumerate}
\end{tcolorbox}
\begin{enumerate}
    \item[a)] $R_A=0.7R_1+0.3R_2$, $R_B=R_1$, тогда $\matwait{R_B}=\matwait{R_1}=0.15$ и  
    \begin{equation*}
        \begin{aligned}
            \matwait{R_A}&=\matwait{0.7R_1+0.3R_2}\\
            &=\matwait{0.7R_1}+\matwait{0.3R_2}\\
            &=0.7\matwait{R_1}+0.3\matwait{R_2}\\
            &=0.15
        \end{aligned}
    \end{equation*}
    Итого, $\matwait{R_A}=\matwait{R_B}=0.15$

    \item[b)] Рассчитаем дисперсии каждого портфеля:
    \begin{itemize}
        \item $\dispersia{R_B}=\dispersia{R_1}=0.49$
        \item \begin{equation*}
            \begin{aligned}
                \dispersia{R_A}&=\dispersia{0.7R_1+0.3R_2}\\
                &=\dispersia{0.7R_1}+\dispersia{0.3R_2}+2cov(0.7R_1,0.3R_2)\\
                &=0.49\dispersia{R_1}+0.09\dispersia{R_2}+2cov(0.7R_1,0.3R_2)\\
                &=0.49\dispersia{R_1}+0.09\dispersia{R_2}+2\cdot0.7\cdot0.3cov(R_1,R_2)\\
                &=0.1831
            \end{aligned}
        \end{equation*}
    \end{itemize}

    \item[c)] \begin{equation*}
        \begin{aligned}
            cov(0.7R_1+0.3R_2,R_1)&=cov(0.7R_1,R_1)+cov(0.3R_2,R_1)\\
            &=0.7\dispersia{R_1}+0.3\cdot cov(R_2,R_1)\\
            &=0.238
        \end{aligned}
    \end{equation*}

    \item[d)] Пусть $\alpha$ — доля $R_1$ и $(1-\alpha)$ — доля $R_2$ в портфеле $C$. Рассчитаем дисперсию нового портфеля

    \begin{equation*}
        \begin{aligned}
            \dispersia{\alpha R_1+(1-\alpha)R_2}&=\alpha^2\dispersia{R_1}+(1-\alpha)^2\dispersia{R_2}+2\alpha(1-\alpha)cov(R_1,R_2)\\
            &=\alpha^2\cdot0.49+(1-\alpha)^2\cdot 1+2(\alpha(1-\alpha)\cdot(-0.35))
            &=\alpha^2(0.49+1+0.7)+\alpha(-2-0.7)+C
        \end{aligned}
    \end{equation*}
    Чтобы добиться наименьшего квадратичного риска нужно минимизировать эту функцию от $\alpha$, возьмем производную:
    \begin{equation*}
        2\cdot2.19\alpha-2.7=0\Longrightarrow\alpha=\frac{2.7}{2\cdot2.19}
    \end{equation*}
\end{enumerate}



\newpage
\section*{№4}
\begin{tcolorbox}[colback=blue!20!white, colframe=black!100!black]
    Величины $X$ и $Y$, описывающие расходы семейной пары, равномерно распределены в треугольнике с вершинами $(0,0),(0,1)$ и $(1,0)$

    \begin{itemize}
        \item[a)] Найдите $\mathbb{P}\left(X^2+Y^2>0.25\right)$
        \item[b)] Найдите частные фукнции плотности и математические ожидания величин $X$ и $Y$
        \item[c)] Проверьте независимость величин $X$ и $Y$
        \item[d)] Вычислите ковариацию величин $X$ и $Y$  
    \end{itemize}
\end{tcolorbox}

\begin{enumerate}
    \item[a)] $\prob{X^2+Y^2>0.25}=\prob{\sqrt{X^2+Y^2}>0.5}=1-\prob{\sqrt{X^2+Y^2}\leqslant0.5}1-\frac{\frac{\pi}{16}}{\frac{1}{2}}=1-\frac{\pi}{8}$

    \item[b)] $\prob{X\leqslant x}=\frac{\frac{(1-x)+1}{2}\cdot x}{\frac{1}{2}}=x(2-x),x\in[0;1]$
    
    Аналогично, для случайной величины $Y$

    Тогда, продифференцировав функцию распределения получим \begin{equation*}f_{X}=\begin{cases}
        2-2x,&x\in[0;1]\\
        0,&\text{иначе}
    \end{cases},\quad f_{Y}=\begin{cases}
        2-2y,&y\in[0;1]\\
        0,&\text{иначе}
    \end{cases}
    \end{equation*}

    \begin{equation*}
        \begin{aligned}
            \matwait{X}&=\int\limits_0^1 xf_{X}(x)\d{x}\\
            \matwait{Y}&=\int\limits_0^1 yf_{Y}(y)\d{y}
        \end{aligned}
    \end{equation*}

    \item[c)] $f_{X,Y}\overset{?}{=}f_{X}(x)\cdot f_{Y}(y)$

    $F_{X,Y}=\prob{X\leqslant x, Y\leqslant y}=\begin{cases}
        2xy,&\begin{aligned}
            x+y\leqslant 1\\
            x\in[0;1]\\
            y\in[0;1]
        \end{aligned}\\
        \frac{S}{\frac{1}{2}},&\begin{aligned}
            x+y>1\\
            x\in[0;1]\\
            y\in[0;1]
        \end{aligned},\text{ где $S$ — площадь части, попавшей в треугольник}
    \end{cases}$
\end{enumerate}



\newpage
\section*{№5}
\begin{tcolorbox}[colback=blue!20!white, colframe=black!100!black]
    Рассмотрим матрицу
    \begin{equation*}
        A=\begin{pmatrix}
            4&b\\
            a&9
        \end{pmatrix}
    \end{equation*}
    \begin{itemize}
        \item[a)] Каким условиям должны удовлетворять константы $a$ и $b$, чтобы матрица $A$ была ковариационной матрицей некоторого случайного вектора?
        \item[b)] Известно, что $A$ — ковариационная матрица вектора $(X, Y)$ и $a b=36$. Каким соотношением связаны компоненты случайного вектора? 
    \end{itemize}
\end{tcolorbox}

\begin{enumerate}
    \item[a)] Должна удовлетворять двум условиям:
    \begin{itemize}
        \item $a=b$
        \item Все главные миноры неотрицательные
    \end{itemize}
    $\det A=4\cdot9-ab\geqslant0\Longrightarrow 36\geqslant ab$

    \item[b)] $ab=36$ и так как $a=b$, то либо $a=b=6$ либо $a=b=-6$
\end{enumerate}




\newpage
\section*{№6}
\begin{tcolorbox}[colback=blue!20!white, colframe=black!100!black]
    Юрий Долгорукий проводит уличный опрос и спрашивает москвичей: «Знаете ли Вы год основания Москвы?» С помощью неравенства Чебышёва определите, сколько нужно опросить человек, чтобы с вероятностью не менее 0.9 доля знающих ответ среди опрошенных отличалась бы от истинной вероятности не более, чем на $0.05$?
\end{tcolorbox}

Выпишем неравенство Чебышёва:
\begin{equation*}
    \prob{|X-\matwait{X}|\geqslant\varepsilon}\leqslant\frac{\dispersia{X}}{\varepsilon^2}
\end{equation*}
Тогда, в нашем случае $\varepsilon=0.05$

Теперь проведем такую цепочку преобразований
\begin{equation*}
    \begin{aligned}
        -\prob{|X-\matwait{X}|\geqslant\varepsilon}&\geqslant-\frac{\dispersia{X}}{\varepsilon^2}\\
        1-\prob{|X-\matwait{X}|\geqslant\varepsilon}&\geqslant1-\frac{\dispersia{X}}{\varepsilon^2}\\
        \prob{|X-\matwait{X}|\leqslant\varepsilon}&\geqslant1-\frac{\dispersia{X}}{\varepsilon^2}
    \end{aligned}
\end{equation*}

Введем случайные величины $Z_1,\ldots,Z_n\sim Bi(1,p)$, где $Z_i$ — ответ $i$-го москвича на вопрос, тогда
\begin{equation*}
    X=\frac{Z_1+Z_2+\ldots+Z_n}{n}\Longrightarrow\matwait{X}=\matwait{Z_i}=p
\end{equation*} 

По свойствам биномиального распределения: $\dispersia{X}=\displaystyle\frac{\dispersia{Z_i}}{n}=\frac{p(1-p)}{n}$

\begin{equation*}
    \begin{aligned}
        1-\frac{p(1-p)}{n}&\geqslant0.9\\
        \frac{p(1-p)}{n}&\leqslant\varepsilon^2\\
        \frac{p-p^2}{n}&\leqslant\frac{\frac{1}{4}}{n}\leqslant0.1\varepsilon^2\\
        n&\geqslant \frac{1}{0.4\varepsilon^2}=\frac{1}{0.4\cdot0.05^2}=1000
    \end{aligned}
\end{equation*}

Получается, что надо опросить хотя бы 1000 человек



\newpage
\section*{№7}
\begin{tcolorbox}[colback=blue!20!white, colframe=black!100!black]
    Второй начальный момент неотрицательной случайной величины $\xi$ равен 3

    Оцените сверху вероятность $\mathbb{P}(\xi \geqslant 3)$
\end{tcolorbox}

Из условия получаем, что $\xi>0, \matwait{\xi^2}=3$, а $\varepsilon=3$

$\prob{\xi\geqslant \varepsilon}\leqslant\frac{\matwait{\xi}}{\varepsilon}\leqslant \frac{\sqrt{3}}{3}=\frac{1}{\sqrt{3}}$

$\dispersia{\xi}=\matwait{\xi^2}-(\matwait{\xi})^2\geqslant 0$

$\matwait{\xi^2}\geqslant (\matwait{\xi})^2$

$3\geqslant(\matwait{\xi})^2\sqrt{3}\geqslant\matwait{\xi}$

\section*{№8}
\begin{tcolorbox}[colback=blue!20!white, colframe=black!100!black]
    Величина $\xi$ имеет плотность распределения
\begin{equation*}
    f(x)=\frac{\exp (-x)}{(1+\exp (-x))^2}
\end{equation*}

Для величины $\xi$ вычислите математическое ожидание, медиану, моду и начальный момент порядка $2023$
\end{tcolorbox}

Имеем $f_{\xi}(x)=\displaystyle\frac{e^{-x}}{(1+e^{-x})^2}$. Напомним, что 
\begin{equation*}
    \matwait{\xi}=\int_{-\infty}^{+\infty}x f_{\xi}(x)\d{x}
\end{equation*}

Основная идея задачи в том, что функция плотности четная, проверим
\begin{equation*}
    \begin{aligned}
        f_{\xi}(-x)&=\frac{e^x}{(1+e^x)^2}\\
        &=\frac{e^x}{1+2e^x+e^{2x}e^{-2x}}\frac{e^{-x}}{e^{-2x}+2e^{-x}+1}\\
        &=\frac{e^{-x}}{(1+e^{-x})^2}=f_{\xi}(x)
    \end{aligned}
\end{equation*}
Получаем, что $f_{\xi}(x)$ — четная

% дополню, когда запись будет)
% Тогда, для вычисления $\matwait{\xi^{2023}}$ нам не нужно считать какие-то огромные степени, а можно просто указать, что так как функция плотности четная, то 
% $q_{0.5}$

% $\int_{-\infty}^{+\infty}f_{\xi}(x)dx=1$

% $\int_{-\infty}^{0}f_{\xi}(x)dx+\int_0^{+\infty}f_{\xi}(x)dx=1$

% $mode=\underset{x}{argmax} f_{\xi}(x)=0$






\end{document}
