\documentclass[a4paper, 10pt]{article}
\usepackage{header}

\title{\LARGE{Теория вероятностей и математическая статистика—2}}
\author{Винер Даниил  \href{https://t.me/danya_vin}{@danya\_vin}}
\date{Версия от \today}
\begin{document}
\maketitle
\tableofcontents
\setlength{\parindent}{15pt}
\setlength{\parskip}{2mm}
\setlist[itemize]{left=1cm}
\setlist[enumerate]{left=1cm}
\newpage
\section{Закон больших чисел. Центральная предельная теорема}
\subsection{Закон больших чисел в форме Бернулли}
Пусть имеются некоторые случайные величины $\xi_i=\begin{cases}
    1,&p\\
    0,&1-p
\end{cases}$, где $p$ — вероятность, что какое-то событие произошло. Тогда $\matwait{\xi}=p$, $\dispersia{\xi}=p(1-p)\leqslant \frac{1}{4}$

\theorem Пусть $\hat{p}=\frac{1}{n}\sum_{i=1}^{n} \xi_i$ — доля успехов в $n$ испытаниях Бернулли, тогда $\hat{p}\overset{p}{\longrightarrow} p$

\proof Распишем по неравенству Чебышёва: 
\begin{equation*}
    \prob{|\hat{p}-p|\geqslant\ve}\leqslant\frac{p(1-p)}{n\ve^2}\leqslant \frac{1}{4n\ve^2}\underset{n\to\infty}{\longrightarrow}0
\end{equation*}

\subsection*{Пример}
Пусть 87\% новорожденных доживают до 50 лет. Тогда $p=0,87$ — вероятность дожить до 50. Рассмотрим $n=1000$ новорожденных

Опредедлим с какой вероятностью данная случайная величина отклонится от своего математического ожидания не более, чем на $0,04$ — $\prob{|\hat{p}-0,87|\leqslant 0,04}$. По Чебышёву:
\begin{equation*}
    \prob{|\hat{p}-p|\leqslant 0,04}\geqslant1-\frac{\dispersia{\hat{p}}}{(0,04)^2}=1-\frac{0,87\cdot0,13}{0,0016\cdot1000}=0,929
\end{equation*}

\subsection{Центральная предельная теорема}
Рассмотрим сумму независимых одинаково распределенных случайных величин:
\begin{equation*}
    S_n=\xi_1+\ldots+\xi_n,
\end{equation*}
при этом существует $\dispersia{\xi_i}\leqslant c$, $\matwait{\xi_i}=\mu$, $\dispersia{\xi_n}=\sigma^2$

Тогда, $Z_n=\displaystyle\frac{S_n-n\mu}{\sqrt{n\sigma^2}}\overset{d}{\longrightarrow}Z$, где $Z\sim \mathcal{N}(0;1)$ — имеет стандартное нормальное распределение

Функция плотности:
\begin{equation*}
    \varphi(z)=\frac{1}{\sqrt{2\pi}}e^{\frac{-z^2}{2}}
\end{equation*}


\subsection{Теорема Муавра-Лапласа}
\theorem Имеется $\xi_i=\begin{cases}
    1,&p\\
    0,&1-p
\end{cases}$. $S_n=\sum \xi_i$ — число успехов в $n$ испытаниях. Тогда
\begin{equation*}
    Z_n=\frac{S_n-np}{\sqrt{np(1-p)}}\overset{d}{\longrightarrow}Z\sim \mathcal{N}(0;1)
\end{equation*}



\subsection*{Пример}
Проходит суд над Бенджамином Споком. Из 300 человек 90 — женщины, которые симпатизируют Споку, при этом 12 присяжных будут судить Спока. Требуется определить мог ли отбор присяжных быть случайным.

Число успехов в данном случае — число женщин среди 300 присяжных. Будем считать, что $p=0.5$, то есть половина женщин.

\begin{equation*}
    \prob{\frac{S_{300}-150}{\sqrt{0.5\cdot0.5\cdot300}}\leqslant \frac{90-150}{\sqrt{75}}}\simeq\Phi(-6.93)\simeq2.3\cdot10^{-12}
\end{equation*}

Значит, практически невозможно случайным образом выбрать 90 или меньше женщин среди 300 присяжных при справедливом распределении, то есть отбор был предвзятым


\subsection{Неравенство Берри-Эссена}
\begin{equation*}
    |F_n-\Phi|\leqslant\frac{C_0\cdot\matwait{|\xi_1-\mu|^3}}{\sigma^3\sqrt{n}},\text{ где }\begin{cases}
        F_n\text{ — функция распределения стандартизированной СВ}\\
        C_0\text{ — константа}\\
        \matwait{|\xi_1-\mu|^3}\text{ — третий абсолютный центральный момент}
    \end{cases}
\end{equation*}



\subsection*{Пример}
Пусть имеется $n=1000$ заключенных договоров страхования с 1 января на 1 год. С вероятностью $p=0.05$ произойдет страховой случай, выплаты по каждому договору — 2000 у.е. $R$ — резерв страховой компании

Требуется определить какой должен быть размер резерва, чтобы страховая компания выполнила свои обязательства с вероятностью $0.99$

$S_n=2000(\xi_1+\ldots+\xi_n)$, $\xi_i\sim Bi(p=0.05)$

\begin{equation*}
    \prob{S_n\leqslant R}=\prob{\frac{\sum\xi_i-0.05\cdot1000}{\sqrt{1000\cdot0.05\cdot0.95}}\leqslant \frac{\frac{R}{2000}-0.05\cdot1000}{\sqrt{1000\cdot0.05\cdot0.95}}}\geqslant 0,99
\end{equation*}
Значит, требуется найти квантиль уровня $0.99$. Он равен $2.33$, тогда
\begin{equation*}
    \frac{\frac{R}{2000}-0.05\cdot1000}{\sqrt{1000\cdot0.05\cdot0.95}}=2.33\Longrightarrow R=132117
\end{equation*}
То есть, для покрытия 99\% страховых случаев у страховой компании резерв должен быть размером $132117$ у.е. Напротив, для покрытия всех случаев $R=2000000$




\newpage
\section{Многомерное нормальное распределение}
\subsection{Одномерное нормальное распределение}
\definition Случайная величина имеет нормальное распределение $X\sim\mathcal{N}(\mu,\sigma^2)$, если функция плотности равна
\begin{equation*}
    f_X(x)=\frac{1}{\sqrt{2\pi}\sigma}e^{-\frac{1}{2}\left(\frac{x-\mu}{\sigma}\right)^2}
\end{equation*}


\subsection{Многомерное нормальное распределение}
\definition Пусть случайные величины $z_1,\ldots,z_n$ независимы и $\sim N(0,1)$. Тогда $z=\begin{pmatrix}
    z_1\\
    \vdots\\
    z_n
\end{pmatrix}$ имеет многомерное нормальное распределение $N(0,I)$, где $I$ — единичная матрица

Функция плотности:
\begin{equation*}
    f_Z(z)=\frac{1}{(\sqrt{2\pi})^n}e^{-\frac{1}{2}\sum_{i=1}^n z_i^2}=\frac{1}{(\sqrt{2\pi})^n}e^{-0.5Z^TZ}
\end{equation*}


\comment Пусть $Z\sim N(0,I)$, $A\in\text{Mat}_{k\times n}$ — матрица полного ранга и $k<n$, то есть rank$A=k$. Тогда
\begin{equation*}
    Y=AZ+b\sim N(b,AA^T)
\end{equation*}
\begin{equation*}
    \begin{aligned}
        f_Y(y)&=\frac{1}{|\det A|}f_Z(A^{-1}(y-b))\\
        &=\frac{1}{(\sqrt{2\pi})^n}\frac{1}{|\det A|}e^{-0.5(y-b)^T(A^{-1})^TA^{-1}(y-b)}\\
        &\text{пусть }AA^T=C\\
        &=\frac{1}{(\sqrt{2\pi})^n}\frac{1}{\sqrt{|C|}}e^{-0.5(y-b)^TC^{-1}(y-b)}
    \end{aligned}
\end{equation*}


\begin{definition}
    Случайная величина $Y\sim N(b,C)$, если
    \begin{equation*}
        f_Y(y)=\frac{1}{(\sqrt{2\pi})^n}\frac{1}{\sqrt{|C|}}e^{-0.5(y-b)^TC^{-1}(y-b)}
    \end{equation*}
\end{definition}

\begin{definition}
    Случайный вектор $Y\sim N(0,C)$, если $\forall a_1,\ldots,a_n$
    \begin{equation*}
        a_1Y_1+a_2Y_2+\ldots+a_nY_n
    \end{equation*}
    либо $N(0,\dot)$ либо const
\end{definition}


\subsection{Свойства многомерного нормального распределения}
Пусть $Y\sim N(b,C)$
\begin{enumerate}
    \item $\matwait{Y}=b,cov(Y)=C$
    
    \proof $Y=AZ+b$, $Z\sim N(0,I)$
    \begin{equation*}
        \begin{aligned}
            cov(Y)=\matwait{(AZ+b-\matwait{AZ+b})(AZ+b-AEZ-b)^T}=AcovZA^T=AA^T=C
        \end{aligned}
    \end{equation*}
    \item Любое линейное невырожденное преобразование многомерного нормаьного дает многомерный нормальный вектор
    
    $\forall B,a:$ $BY+a\sim N(Bb+a,BCB^T)$

    \item $\forall$ подвектор нормального вектора нормален 
    \item Если $Y\sim N(b,D)$, то его компоненты независимы
    
    \comment Некоррелированность = независимость

    \proof 
    \begin{equation*}
        \begin{aligned}
            f_Y(y)&=\frac{1}{(\sqrt{2\pi}^n)}e^{-0.5(y-b)^T D^{-1}(y-b)}\\
            &=\frac{1}{(\sqrt{2\pi}^n)}e^{-0.5\sum \left(\frac{y_i-b_i}{\sigma_i}\right)^2}\\
            &=\prod_{i=1}^n\frac{1}{\sqrt{2\pi}}e^{-0.5\left(\frac{y_i-b_i}{\sigma_i}\right)^2}
        \end{aligned}
    \end{equation*}
\end{enumerate}

\ex $Y_1\sim N(0,1),\ \lambda=\begin{cases}
    1,&p=0.5\\
    -1,&p=0.5
\end{cases},\ Y_2=2Y_1$

\begin{equation*}
    \begin{aligned}
        \prob{Y_2\leqslant y}&=\prob{Y_1\leqslant y|\alpha=1}\cdot\prob{\alpha=1}+\prob{-Y_1\leqslant y|\alpha=-1}\cdot\frac{1}{2}\\
        &=\Phi(y)
    \end{aligned}
\end{equation*}

$cov(Y_1,Y_2)=cov(Y_1,2Y_1)=\matwait{\alpha Y_1^2}-\matwait{Y}\matwait{\alpha Y_1}=0$. То есть они не коррелированы


\subsection{Условное нормальное распределение}
Имеется случайный вектор $\begin{pmatrix}
    z_1\\
    z_2
\end{pmatrix}\sim N\left(\begin{pmatrix}
    0\\
    0
\end{pmatrix},\begin{pmatrix}
    1&\rho\\
    \rho&1
\end{pmatrix}\right)$, пишут $\Phi_2(z_1,z_2;\rho)$

Допустим, что $z_2$ фиксирован, тогда $z_2|z_1=z\sim N(\rho z,1-\rho^2)$

$z_2=\rho z_1+u$, где $z_1$ и $u$ независимы и $u\sim N(.,.)$





\end{document}













