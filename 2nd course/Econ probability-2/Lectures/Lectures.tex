\documentclass[a4paper, 10pt]{article}
\usepackage{header}

\title{\LARGE{Теория вероятностей и математическая статистика—2}}
\author{Винер Даниил  \href{https://t.me/danya_vin}{@danya\_vin}}
\date{Версия от \today}
\begin{document}
\maketitle
\tableofcontents
\setlength{\parindent}{15pt}
\setlength{\parskip}{2mm}
\setlist[itemize]{left=1cm}
\setlist[enumerate]{left=1cm}
\newpage
\section{Лекция 13.01.2025}
\subsection{Закон больших чисел в форме Бернулли}
Пусть имеются некоторые случайные величины $\xi_i=\begin{cases}
    1,&p\\
    0,&1-p
\end{cases}$, где $p$ — вероятность, что какое-то событие произошло. Тогда $\matwait{\xi}=p$, $\dispersia{\xi}=p(1-p)\leqslant \frac{1}{4}$

\theorem Пусть $\hat{p}=\frac{1}{n}\sum_{i=1}^{n} \xi_i$ — доля успехов в $n$ испытаниях Бернулли, тогда $\hat{p}\overset{p}{\longrightarrow} p$

\proof Распишем по неравенству Чебышёва: 
\begin{equation*}
    \prob{|\hat{p}-p|\geqslant\ve}\leqslant\frac{p(1-p)}{n\ve^2}\leqslant \frac{1}{4n\ve^2}\underset{n\to\infty}{\longrightarrow}0
\end{equation*}

\subsection*{Пример}
Пусть 87\% новорожденных доживают до 50 лет. Тогда $p=0,87$ — вероятность дожить до 50. Рассмотрим $n=1000$ новорожденных

Опредедлим с какой вероятностью данная случайная величина отклонится от своего математического ожидания не более, чем на $0,04$ — $\prob{|\hat{p}-0,87|\leqslant 0,04}$. По Чебышёву:
\begin{equation*}
    \prob{|\hat{p}-p|\leqslant 0,04}\geqslant1-\frac{\dispersia{\hat{p}}}{(0,04)^2}=1-\frac{0,87\cdot0,13}{0,0016\cdot1000}=0,929
\end{equation*}

\subsection{Центральная предельная теорема}
Рассмотрим сумму независимых одинаково распределенных случайных величин:
\begin{equation*}
    S_n=\xi_1+\ldots+\xi_n,
\end{equation*}
при этом существует $\dispersia{\xi_i}\leqslant c$, $\matwait{\xi_i}=\mu$, $\dispersia{\xi_n}=\sigma^2$

Тогда, $Z_n=\displaystyle\frac{S_n-n\mu}{\sqrt{n\sigma^2}}\overset{d}{\longrightarrow}Z$, где $Z\sim \mathcal{N}(0;1)$ — имеет стандартное нормальное распределение

Функция плотности:
\begin{equation*}
    \varphi(z)=\frac{1}{\sqrt{2\pi}}e^{\frac{-z^2}{2}}
\end{equation*}


\subsection{Теорема Муавра-Лапласа}
\theorem Имеется $\xi_i=\begin{cases}
    1,&p\\
    0,&1-p
\end{cases}$. $S_n=\sum \xi_i$ — число успехов в $n$ испытаниях. Тогда
\begin{equation*}
    Z_n=\frac{S_n-np}{\sqrt{np(1-p)}}\overset{d}{\longrightarrow}Z\sim \mathcal{N}(0;1)
\end{equation*}



\subsection*{Пример}
Проходит суд над Бенджамином Споком. Из 300 человек 90 — женщины, которые симпатизируют Споку, при этом 12 присяжных будут судить Спока. Требуется определить мог ли отбор присяжных быть случайным.

Число успехов в данном случае — число женщин среди 300 присяжных. Будем считать, что $p=0.5$, то есть половина женщин.

\begin{equation*}
    \prob{\frac{S_{300}-150}{\sqrt{0.5\cdot0.5\cdot300}}\leqslant \frac{90-150}{\sqrt{75}}}\simeq\Phi(-6.93)\simeq2.3\cdot10^{-12}
\end{equation*}

Значит, практически невозможно случайным образом выбрать 90 или меньше женщин среди 300 присяжных при справедливом распределении, то есть отбор был предвзятым


\subsection{Неравенство Берри-Эссена}
\begin{equation*}
    |F_n-\Phi|\leqslant\frac{C_0\cdot\matwait{|\xi_1-\mu|^3}}{\sigma^3\sqrt{n}},\text{ где }\begin{cases}
        F_n\text{ — функция распределения стандартизированной СВ}\\
        C_0\text{ — константа}\\
        \matwait{|\xi_1-\mu|^3}\text{ — третий абсолютный центральный момент}
    \end{cases}
\end{equation*}



\subsection*{Пример}
Пусть имеется $n=1000$ заключенных договоров страхования с 1 января на 1 год. С вероятностью $p=0.05$ произойдет страховой случай, выплаты по каждому договору — 2000 у.е. $R$ — резерв страховой компании

Требуется определить какой должен быть размер резерва, чтобы страховая компания выполнила свои обязательства с вероятностью $0.99$

$S_n=2000(\xi_1+\ldots+\xi_n)$, $\xi_i\sim Bi(p=0.05)$

\begin{equation*}
    \prob{S_n\leqslant R}=\prob{\frac{\sum\xi_i-0.05\cdot1000}{\sqrt{1000\cdot0.05\cdot0.95}}\leqslant \frac{\frac{R}{2000}-0.05\cdot1000}{\sqrt{1000\cdot0.05\cdot0.95}}}\geqslant 0,99
\end{equation*}
Значит, требуется найти квантиль уровня $0.99$. Он равен $2.33$, тогда
\begin{equation*}
    \frac{\frac{R}{2000}-0.05\cdot1000}{\sqrt{1000\cdot0.05\cdot0.95}}=2.33\Longrightarrow R=132117
\end{equation*}
То есть, для покрытия 99\% страховых случаев у страховой компании резерв должен быть размером $132117$ у.е. Напротив, для покрытия всех случаев $R=2000000$









\end{document}













