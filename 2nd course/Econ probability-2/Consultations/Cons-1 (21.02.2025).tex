\documentclass{article}
\usepackage{header} % Required for inserting images
% \newcommand{\p}[1]{$\mathbb{P}(#1)$}


\title{\LARGE{Теория вероятностей и математическая статистика—2}\\
Консультация Борзых Д.А.\\
ФЭН НИУ ВШЭ}
\author{Винер Даниил  \href{https://t.me/danya_vin}{@danya\_vin}}
\date{21 февраля 2025г.}

\begin{document}
\maketitle
% \tableofcontents
% \newpage
\setlength{\parindent}{15pt}
\setlength{\parskip}{2mm}
\setlist[itemize]{left=1cm}
\setlist[enumerate]{left=1cm}
\section*{23} 
\definition Говорят, что случайная величина $X$ имеет нормальное распределение с параметрами $\mu\in\mathbb{R}$ и $\sigma^2>0$ пишут $X\sim N(\mu,\sigma^2)$, если
\begin{equation*}
    f_X(x)=\frac{1}{\sqrt{2\pi\sigma^2}}\exp\left(-\frac{(x-\mu)^2}{2\sigma^2}\right)
\end{equation*} 

Множество значений случайной величины $X$: $(-\infty;+\infty)$

\theorem Если $X\sim N(\mu,\sigma^2)$, то 
\begin{equation*}
    \matwait{X}=\mu,\ \dispersia{X}=\sigma^2
\end{equation*}


\section*{24}
\definition Случайная величина $W$ имеет $\chi^2$-распределение с $m$ степенями свободы, пишут\\
$W\sim\chi^2(m)$, если $W$ представима в виде
\begin{equation*}
    W=X_1^2+\ldots+X_m^2,
\end{equation*}
где $X_1,\ldots,X_n\sim iidN(0;1)$ 

Множество значений случайной величины $W$: $(0;+\infty)$

\theorem Если $W\sim\chi^2(m)$, то $\matwait{W}=m$, $\dispersia{W}=2m$


\section*{25}
\definition Случайная величина $W$ имеет $t$-распределение (распределение Стьюдента) с $m$ степенями свободы, пишут $W\sim t(m)$, если $W$ представима в виде
\begin{equation*}
    W=\frac{X}{\sqrt{\frac{Y_1^2+\ldots+Y_m^2}{m}}},
\end{equation*}
где $X,Y_1,\ldots,Y_m\sim iidN(0;1)$

Множество значений случайной величины: $(-\infty;+\infty)$

\section*{26}
\definition Случайная величина $W$ имеет $F$-распределение (распределение Фишера) с $m$ и $n$ степенями свободы, пишут $W\sim F(m,n)$, если $W$ представима в виде
\begin{equation*}
    W=\frac{(X_1+\ldots+X_m^2)/m}{(Y_1^2+\ldots+Y_n^2)/n},
\end{equation*}
где $X_1,\ldots,X_m,Y_1,\ldots,Y_n\sim iidN(0;1)$ и независимы

Множество значений: $(0;+\infty)$

\section*{27}
Пусть $X=(X_1,\ldots,X_n)$ — случайная выборка

\definition Выборочное среднее — $\overline{X}:=\displaystyle\frac{X_1+\ldots+X_n}{n}$

\definition Неисправленная выборочная дисперсия — $s^2:=\displaystyle\frac{1}{n}\sum_{i=1}^n (X_i-\overline{X})^2$

% \definition Исправленная выборочная дисперсия — $\widehat{\sigma}^2:=\displaystyle\frac{1}{n-1}\sum_{i=1}^n (X_i-\overline{X})^2$

\section*{28}
Пусть $X=(X_1,\ldots,X_n)$ — случайная выборка, тогда 

\definition Выборочным \textit{начальным} моментом порядка $k$ называется число 
\begin{equation*}
    \widehat{\mu}_k:=\frac{1}{n}\sum_{i=1}^{k}(X_i)^k
\end{equation*}

\definition Выборочным \textit{центральным} моментом порядка $k$ называется число
\begin{equation*}
    \widehat{\nu}_k:=\frac{1}{n}\sum_{i=1}^{k}(X_i-\overline{X})^k
\end{equation*}


\section*{29}
\definition Пусть $X=(X_1,\ldots,X_n)$ — случайная выборка, тогда выборочной функцией распределения случайной выборки $X$ называется функция от действительного переменного $X$, которая определяется как
\begin{equation*}
    \widehat{F}_n(x,\omega):=\frac{1}{n}\sum_{i=1}^{n}\mathbb{I}\{X_i(\omega)\leqslant x\},
\end{equation*}
где $\mathbb{I}\{X_i(\omega)\leqslant x\}$ равна 1, если $X_i(\omega)\leqslant x$ и равна 0 в противном случае

\section*{30}
\theorem Пусть $X=(X_1,\ldots,X_n)$ — случайная выборка, причем $\dispersia{X_i}=\sigma^2$, тогда 
\begin{equation*}
    \widehat{\sigma}^2:=\displaystyle\frac{1}{n-1}\sum_{i=1}^n (X_i-\overline{X})^2
\end{equation*}
является несмещённой оценкой параметра $\sigma^2$, то есть $\matwait{\widehat{\sigma}^2}=\sigma^2$

\section*{31}
Оценка $\hat{\theta}$ неизвестного параметра $\theta\in\Theta$ называется \textit{несмещённой}, если 
\begin{equation*}
    \matwait{\hat{\theta}}=\theta\ \forall\theta\in\Theta,
\end{equation*}
где $\Theta$ — множество всех допустимых значений параметра $\theta$

\section*{32}
\definition Последовательность оценок $\hat{\theta}_n$ называется \textit{состоятельной оценкой} неизвестного параметра $\theta\in\Theta$, если
\begin{equation*}
    \forall \theta\in\Theta\ \hat{\theta}_n\underset{n\to\infty}{\overset{\mathbb{P}}{\longrightarrow}}\theta,
\end{equation*}
то есть $\hat{\theta}_n$ сходится по вероятности к $\theta$
\begin{equation*}
    \forall\varepsilon>0:\ \lim\limits_{n\to\infty}\mathbb{P}(|\hat{\theta}_n-\theta|\geqslant\varepsilon)=0
\end{equation*}


\theorem Пусть 
\begin{itemize}
    \item $forall\theta\Theta\ \matwait{\widehat{\theta}_n}\underset{n\to\infty}{\longrightarrow}\theta$
    \item $forall\theta\Theta\ \dispersia{\widehat{\theta}_n}\underset{n\to\infty}{\longrightarrow}0$
\end{itemize}
Тогда, $\widehat{\theta}_n\overset{\mathbb{P}}{\longrightarrow}\theta$, то есть $\widehat{\theta}_n$ является состоятельной оценкой  неизвестного параметра $\theta$

\section*{33}
\definition Пусть $\mathcal{K}$ — некоторый класс оценок параметра $\theta$. Оценка $\widehat{\theta}\in\mathcal{K}$ неизвестного параметра $\theta$ называется наиболее эффективной в классе $\mathcal{K}$, если для любого конкурента $\widetilde{\theta}\in\mathcal{K}\ \forall \theta\in\Theta$
\begin{equation*}
    \begin{aligned}
        \matwait{(\widehat{\theta}-\theta)^2}\leqslant\matwait{(\widetilde{\theta}-\theta)^2}
    \end{aligned}
\end{equation*}








\end{document}