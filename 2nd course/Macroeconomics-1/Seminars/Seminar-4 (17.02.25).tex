\documentclass[a4paper, 10pt]{article}
\usepackage{header}

\title{\LARGE{Макроэкономика—1 (углубленный курс)}\\[2mm]
\large{Семинар}}
\author{Винер Даниил  \href{https://t.me/danya_vin}{@danya\_vin}}
\date{17 февраля 2025 г.}
\begin{document}
\maketitle
\setlength{\parindent}{15pt}
\setlength{\parskip}{2mm}
\setlist[itemize]{left=1cm}
\setlist[enumerate]{left=1cm}

\section*{№1}
Пусть $Y=K^{\alpha}(AL)^{1-\alpha}$, $\displaystyle\frac{\dot{A}}{A}=g, \frac{\dot{L}}{L}=n, \dot{K}=sY-\delta K$

\begin{tcolorbox}[colback=blue!20!white, colframe=black!100!black]
    \textbf{a)} Характеризуется ли данная производственная функция постоянной отдачей от масштаба по труду и капиталу? Какая отдача от масштаба у данной производственной функции по труду, капиталу и технологии?
\end{tcolorbox}
Производственная функция обладает постоянной отдачей от масштаба, если 
\begin{equation*}
    F(\lambda K,\lambda AL)=\lambda F(K, AL)
\end{equation*}

В нашем случае, эта производственная функция \textit{обладает} постоянной отдачей от масштаба по труду и капиталу:
\begin{equation*}
    \begin{aligned}
        F(\lambda K,\lambda AL)&=\lambda F(K, AL)\\
        (\lambda K)^{\alpha}(\lambda AL)^{1-\alpha}&=\lambda K^{\alpha}(AL)^{\alpha}\\
        \lambda^{\alpha}K^{\alpha}\lambda^{1-\alpha}A^{1-\alpha}L^{1-\alpha}&=\lambda K^{\alpha}(AL)^{\alpha}\\
        \lambda K^{\alpha}(AL)^{1-\alpha}&=\lambda K^{\alpha}(AL)^{\alpha}
    \end{aligned}
\end{equation*}

Рассмотрим, как изменяется выпуск при пропорциональном увеличении капитала, труда и технологии при $\lambda>0$:
\begin{equation*}
    \begin{aligned}
        F(\lambda K, \lambda A\lambda L)&= (\lambda K)^\alpha (\lambda A \lambda L)^{1-\alpha}\\
        &=\lambda^{2-\alpha}K^{\alpha}(AL)^{1-\alpha}>\lambda K^{\alpha}(AL)^{1-\alpha}
    \end{aligned}
\end{equation*}
Это означает, что функция показывает \textit{возрастающую} отдачу от масштаба

\begin{tcolorbox}[colback=blue!20!white, colframe=black!100!black]
    \textbf{b)} Запишите производственную функцию в интенсивной форме и основное уравнение динамики модели Солоу.
\end{tcolorbox}
\begin{equation*}
    \begin{aligned}
        y&=\frac{Y}{AL}\\
        &=\frac{K^{\alpha}(AL)^{1-\alpha}}{AL}\\
        &=\frac{K^{\alpha}}{(AL)^{\alpha}}\\
        &=\left(\frac{K}{AL}\right)^{\alpha}\\
        &=k^{\alpha}
    \end{aligned}
\end{equation*}
$y=k^{\alpha}$ — производственная функция в \textit{интенсивной} форме

\begin{equation*}
    \begin{aligned}
        \dot{k}&=\dot{\left(\frac{K}{AL}\right)}\\
        &=\frac{\dot{K}\cdot AL-K\cdot \dot{(AL)}}{(AL)^2}\\
        &=\frac{\dot{K}}{AL}-\frac{K}{AL}\cdot\frac{\dot{AL}}{AL}\\
        &=\frac{sy-\delta k}{AL}-\frac{K}{AL}\left(\frac{\dot{A}}{A}+\frac{\dot{L}}{L}\right)\\
        &=sy-\delta k - k(g+n)\\
        &=sy-(n+g+\delta)k
    \end{aligned}
\end{equation*}
$\dot{k}=sy-(n+g+\delta)k$ — основное уравнение динамики модели Солоу

\begin{tcolorbox}[colback=blue!20!white, colframe=black!100!black]
    \textbf{c)} Найдите капиталовооруженность эффективного труда на траектории сбалансированного роста (ТСР). Проверьте, что в модели существует единственно глобально устойчивое стационарное состояние. Найдите выпуск и потребление в расчете на одного эффективного работника на ТСР
\end{tcolorbox}

На ТСР переменная $k$ находится в стационарном состоянии, то есть темп её изменения равен нулю:
\begin{equation*}
    \dot{k}=0
\end{equation*}
Из основного уравнения динамики модели Солоу получаем
\begin{equation*}
    \begin{aligned}
        sk^{\alpha}&=(n+g+\delta)k\quad\vert :k\\
        sk^{\alpha-1}&=n+g_\delta\\
        k^{*}&=\left(\frac{s}{n+g+\delta}\right)^{\frac{1}{1-\alpha}}
    \end{aligned}
\end{equation*}


На ТСР выпуск на одного эффективного работника определяется как:
\begin{equation*}
    \begin{aligned}
        y^{*}&=(k^{*})^{\alpha}\\
        &=\left(\frac{s}{n+g+\delta}\right)^{\frac{\alpha}{1-\alpha}}
    \end{aligned}
\end{equation*}

Потребление на одного эффективного работника:
\begin{equation*}
    \begin{aligned}
        c^{*}&=(1-s)y^{*}\\
        &=(1-s)=\left(\frac{s}{n+g+\delta}\right)^{\frac{\alpha}{1-\alpha}}
    \end{aligned}
\end{equation*}

\begin{tcolorbox}[colback=blue!20!white, colframe=black!100!black]
    \textbf{d)} Найдите капитал $(K)$, выпуск $(Y)$ и потребление $(С)$ на ТСР, а также темпы их роста.
\end{tcolorbox}

\begin{equation*}
    \begin{aligned}
        y&=\frac{Y}{AL}\\
        Y&=y\cdot AL\text{ — прологарифмируем}\\
        \ln Y&=\ln y+\ln A+\ln L\quad\vert\frac{\partial }{\partial t}\\
        \left(\ln Y\right)^{\prime}_t&=\frac{1}{y}\cdot y^{\prime}_t,\text{ при этом} \left(\ln Y\right)^{\prime}_t=\dot{\ln Y}=\frac{\dot{Y}}{Y}\\
        \frac{\dot{Y}}{Y}&=\frac{\dot{y}}{y}+\frac{\dot{A}}{A}+\frac{\dot{L}}{L}\\
        &=g+n
    \end{aligned}
\end{equation*}
То есть, темпы роста ВВП равны $g+n$

По условию
\begin{equation*}
    \begin{aligned}
        \frac{\dot{A}}{A}&=g\\
        \frac{\frac{dA}{dt}}{A}&=g\\
        \int\frac{1}{A}dA&=\int gdt\\
        \ln A&=gt+C\\
        A&=e^{gt}\cdot e^{C}
    \end{aligned}
\end{equation*}
Положим, что в начальный момент времени $A(t=0)=e^c=A_0$, тогда 
\begin{equation*}
    \begin{aligned}
        A&=A_0 e^{gt}
    \end{aligned}
\end{equation*}
Рассуждая аналогично, получим $L=L_0e^{nt}$, тогда выпуск
\begin{equation*}
    \boxed{Y^{*}(t)=\underbrace{\left(\frac{s}{n+g+\delta}\right)^{\frac{\alpha}{1-\alpha}}}_{y^{*}}\cdot\underbrace{A_0e^{gt}}_{A(t)}\cdot\underbrace{L_0e^{nt}}_{L(t)}}
\end{equation*}

Рассмотрим для капитала
\begin{equation*}
    \begin{aligned}
        k&=\frac{K}{AL}\\
        K&=kAL\\
        \ln K&=\ln k+\ln A+\ln L\\
        \frac{\dot{K}}{K}&=\frac{\dot{k}}{k}+\frac{\dot{A}}{A}+\frac{\dot{L}}{L}\\
        &=g+n
    \end{aligned}
\end{equation*}
\begin{equation*}
    \boxed{K^{*}(t)=\left(\frac{s}{n+g+\delta}\right)^{\frac{1}{1-\alpha}}A_0e^{gt}\cdot L_0e^{nt}}
\end{equation*}

Теперь для потребления
\begin{equation*}
    \begin{aligned}
        C&=(1-s)Y\\
        \ln C&=\ln(1-s)+\ln Y\\
        \frac{\dot{C}}{C}&=\frac{\dot{Y}}{Y}\\
        &=n+g
    \end{aligned}
\end{equation*}
Получется, что $\boxed{C^{*}(t)=(1-S)Y^{*}(t)}$














\end{document}