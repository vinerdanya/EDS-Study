\documentclass[a4paper, 10pt]{article}
\usepackage{header}

\title{\LARGE{Макроэкономика—1 (углубленный курс)}\\[2mm]
\large{Семинар}}
\author{Винер Даниил  \href{https://t.me/danya_vin}{@danya\_vin}}
\date{20 января 2025 г.}
\begin{document}
\maketitle
\setlength{\parindent}{15pt}
\setlength{\parskip}{2mm}
\setlist[itemize]{left=1cm}
\setlist[enumerate]{left=1cm}
\section{Методы подсчета ВВП}
\subsection{ВВП по расходам}
\definition Основное макроэкономическое тождество:
\begin{equation*}
    \text{ВВП}=C+I+G+Xn,
\end{equation*}
где $C$ — потребление, $I$ — инвестиции, $G$ — государственные расходы, $Xn=Ex-Im$ — чистый экспорт

Например, в Россию ввезли Audi за $20000\$$, при этом ничего нового в стране не было произведено, то есть ВВП не должен измениться. Тогда, в $Im$ мы добавим $20000\$$. Так как мы вычитаем импорт, то его нужно куда-то прибавить, он пойдет в потребление (consumption). Машина будет продана по цене, большей $20000\$$, например $30000\$$. Тогда, ВВП России увеличится на $10000\$$


\subsection{ВВП по доходам}
\begin{equation*}
    \text{ВВП}=\underbrace{\underbrace{\text{ЗП}+\%+\text{рента}+\text{прибыль}}_{\text{нац.доход}}+Tx_{\text{косвенные}}-\text{subsidies}+\text{amortization}}_{\text{ВНД}}+NFFI
\end{equation*}

Рассмотрим страну $A$, в ней живут и работают иностранные граждане, обозначим их как $B$. Также, чать граждан страны $A$ живет и работает зарубежом. Тогда, верны такие суждения
\begin{equation*}
    \left.\begin{aligned}
        \text{ВВП}&=A+B\\
        \text{ВНД}&=A+C\\
    \end{aligned}\right\}\Longrightarrow\text{ВВП}=\text{ВНД}+\underbrace{B-C}_{NFFI}
\end{equation*}

\definition $NFFI$ — Net Foreign Factor Income или же чистый доход иностранных факторов, то есть доходы иностранных граждан на территории нашей страны

\subsection{ВВП по добавленным стоимостям}
\definition \textit{Добавленная стоимость} = $TR$ — затраты на сырье и материалы

В таком случае
\begin{equation*}
    \text{ВВП}=\sum_i\text{ДС}_i
\end{equation*}

\ex Хлебозавод купил пшеницу за 100р., потом изготовил муку и продал за 120р. пекарне. В ВВП добавится 20р. Пекарня испекла хлеб и продала его за 170р. В ВВП добавится 50р.


\section{Задачи}
\subsection{Задача 1}
В расчете ВВП учитываются пункты
\begin{equation*}
    1,2,3,6,10,11,13
\end{equation*}

При этом, \textit{зарплата, полученная рабочими за работу в другой стране, и пересланная домой} будет учитываться в ВВП той страны, где они работали, но не будет учитываться в её ВНД

\subsection{Задача 2}
\begin{equation*}
    \begin{tabular}{c|c|c}
        &\text{РФ}&\text{США}\\
        \hline
        \text{ВВП}&15000&20000\\
        \hline
        \text{ВНД}&5000+5000&20000+2000+3000
    \end{tabular}
\end{equation*}

\subsection{Задача 3}
Во Франции ВВП удвоится за 35 лет. Тогда, чтобы ВВП на душу населения сравнялся через 35 лет российскому ВВП нужно вырасти в 4 раза, то есть удвоиться ему надо за $\frac{35}{2}=17.5$ лет. Пусть темп роста $x\%$ в год, тогда 
\begin{equation*}
    \frac{70}{x}=\frac{35}{2}\Longrightarrow x=4
\end{equation*}

То есть для РФ темпы роста должны быть $4\%$ в год

\subsection{Задача 4}
Пусть темп роста $x\%$ в год. Значит, 
\begin{equation*}
    (1+x)^T=3\Longrightarrow T=\log_{1+x}3=\frac{\ln(3)}{\ln(1+x)}=\frac{\ln(3)}{x}\approx\frac{1.1}{x}\underset{\text{в проценты}}{\overset{\text{переведем}}{\approx}}\frac{110}{x}\text{ лет}
\end{equation*}




\end{document}